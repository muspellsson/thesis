\documentclass{article}

%\usepackage[
%	a4paper, mag=1000,
%	left=2.5cm, right=1cm, top=2cm, bottom=2cm, headsep=0.7cm, footskip=1cm
%]{geometry}
\usepackage[utf8]{inputenc}
\usepackage[T2A]{fontenc}
\usepackage[english,russian]{babel}
\usepackage{amsmath}
%\ifpdf\usepackage{epstopdf}\fi

%\pagestyle{footcenter}
%\chapterpagestyle{footcenter}

\begin{document}
\title{Множества реализуемых режимов термодинамических систем и выбор порядка разделения многокомпонентных смесей}
\author{Сукин И.А.}
\maketitle

\tableofcontents

\section*{Введение}

Диссертация посвящена разработке методов построения множеств реализуемых режимов для термодинамических систем, состоящих из заданного числа аппаратов. С использованием этих методов построены конкретные алгоритмы управления некоторыми технологическими процессами, такими как ректификация двух- и многокомпонентных смесей, система охлаждения суперкомпьютера.

Имеющее действительный практический интерес решение задачи построения множества реализуемых режимов стало возможным с появлением аппарата оптимизационной термодинамики, или, как часто ее не совсем верно называют, термодинамики при конечном времени. Для этого в рассмотрение были введены необратимые процессы. В классической термодинамике, имеющей дело только с обратимыми процессами, может быть построен только один вид границ области реализуемости --- прямые $g = \eta q$, где $g$ --- производительность, $q$ --- нагрузка, $\eta$ --- коэффициент полезного действия системы.

Для рассматриваемых в работе процессов описаны математические модели и приведены способы построения областей достижимости. Для процесса бинарной ректификации с использованием полученных результатов приведены алгоритмы расчета и уточнения некоторых технологических параметров.

Полученные результаты могут быть использованы для анализа многоаппаратных термодинамических систем, автоматического управления такими системами, уточнения аппарата оптимизационной термодинамики.

\section{Методология оптимизационной термодинамики и построение множеств достижимости}

	\subsection{Термодинамика как научная дисциплина}
	
	Термодинамика как наука занимает особое положение среди остальных областей знания, что обуславливает ее методы и получаемые результаты. Она является универсальной феноменологической дисциплиной, имеющей дело с явлением, процессом в целом, без углубления в его внутреннюю структуру. Это обстоятельство позволяет решать с помощью методов термодинамики широкий класс задач, относящихся к организации различных физических процессов.
	
	Все процессы в термодинамике рассматриваются как процессы преобразования энергии. В таком виде могут быть представлены явления самой разной природы: физические процессы, такие как теплоперенос и массоперенос, процесс, происходящие во время химических реакций, микроэкономические процессы. Универсальность термодинамики позволяет использовать единый аппарат для описания всех систем преобразования энергии. При этом внутренние (технологические) свойства таких процессов описываются законами конкретной науки, эмпирического исследования.
	
	Основным объектом изучения в термодинамике является макросистема --- система, состоящая из большого числа элементов, имеющих неуправляемый стохастический характер. При этом система в целом остается управляемой. Такая специфика макросистем обуславливает широкое применение усредненных методов для решения задач управления.

	\subsection{Основные понятия термодинамики}
	
	\subsubsection{Координата и потенциал}
	
	Двумя важнейшими понятиями в термодинамике являются координата и потенциал. Координатой называют величину, изменение которой свидетельствует о том, что в рассматриваемой системе происходит изучаемое взаимодействие (теплообмен, массообмен, совершение механической работы, деформация, химическая реакция). К примеру, координатой деформационного взаимодействия является объем, координатой химического взаимодействия --- степень полноты реакции.
	
	Потенциал --- это величина, различие в значениях которой является причиной взаимодействия. Для деформации это давление, для химического взаимодействия --- сумма химических потенциалов реагентов.
	
	\subsubsection{Экстенсивные и интенсивные переменные}
	
	Переменные, характеризующие термодинамические системы можно разделить на два класса: интенсивные и экстенсивные. Первые (температура, давление) не изменяются при объединении двух идентичных подсистем, а вторые (объем, число молей) увеличиваются в два раза.
	
	\subsubsection{Равновесие}
	
	О термодинамической системе или ее подсистеме говорят, что она находится в состоянии равновесия, если величины потенциалов каждого рода по ее объему равны. Если каким-либо образом изменить потенциал системы в некоторой точке (внести в нее вещество, энергию, изменить давление), в ней начнет происходить некоторый процесс, сопровождающийся выравниванием значений потенциалов. Это процесс называется релаксацией и требует определенного времени (это замечание не столь тривиально, как может показаться).
	
	Если различие потенциалов в подсистеме мало, внешние условия изменяются медленно, а скорость их изменения значительно меньше скорости релаксации подсистемы, поэтому в каждый момент времени ее состояние будет мало отличаться от равновесного. Такие процессы называют равновесными, время их протекания бесконечно велико. Чаще всего можно считать, что равновесные процессы обратимы, то есть подсистему можно вернуть из конечного состояния в начальное так, что переменные, характеризующие ее состояние будут принимать тот же ряд значений, что и при прямом процессе, но в обратном порядке, а обмен веществом или энергией с другими подсистемами имеет ту же величину, что и в прямом процессе, но обратный знак. Некоторые реальные процессы (теплообмен, массообмен, обратимые реакции) можно проводить бесконечно близко к обратимым, в то время как другие (смешение жидкостей или газов) --- нельзя, поскольку для проведения их обратном порядке придется затратить дополнительную работу. Равновесные процессы рассматриваются в классической термодинамике.
	
	В реальности все процессы протекают с некоторой конечной, а не бесконечно малой скоростью. Состояние подсистем при это значительно отличается от равновесного, а сами процессы являются необратимыми. При рассмотрении таких процессов обычно принято исходить из предположения, что каждую подсистему можно разбить на бесконечно малые области, в рамках которых состояние системы равновесно, а взаимодействие внутри подсистемы в целом рассматривается как взаимодействие этих бесконечно малых областей. Это предположение называется гипотезой локального равновесия. Рассмотрением неравновесных процессов занимается неравновесная термодинамика.
	
	Методы оптимизационной термодинамики или термодинамики при конечном времени расположены на границе между равновесной и неравновесной термодинамикой. Они рассматривают систему как неравновесную совокупность равновесных подсистем. Это позволяет избежать чрезмерного усложнения математического аппарата, необходимого для описания системы, но в то же время получить необратимые оценки параметров.
	
	\subsubsection{Открытые системы}
	
	Открытой системой в термодинамике называется система, которая обменивается с окружением материальными и энергетическими потоками.
	
	Состоянию равновесия в такой системе соответствует неизменность во времени интенсивных переменных (однако от подсистемы к подсистеме их значения могут меняться). Для функционирования открытой системы в стационарном режиме она должна содержать не менее двух резервуаров --- систем, интенсивные переменные которых можно считать постоянными.
	
	\subsubsection{Замкнутые системы}
	
	Термодинамическая система может быть изолирована от окружения по всем или только по некоторым видам потоков. Такой тип систем называется замкнутыми системами. Равновесие в полностью изолированной замкнутой системе при наличии контакта между подсистемами соответствует равенству значений интенсивных переменных отдельных подсистем.
	
	\subsubsection{Уравнения состояния}
	
	В состоянии равновесия система характеризуется только частью переменных, которые называются независимыми. Остальные переменные находятся через независимые из уравнения состояния. Эти уравнения получают как обработкой эмпирических экспериментальных наблюдений над системой, так и на основе моделирования свойств составляющих систему элементов.
	
	\subsubsection{Термодинамические балансы}
	
	Термодинамические балансы устанавливают связь между потоками по каждому из веществ, а также энергетическими потоками. Первый тип баланса называется материальным балансом, а второй, соответственно, энергетическим. Балансовые уравнения соответствуют законам сохранения. Эти уравнения выделяют в пространстве потоков область, называемую областью реализуемости или достижимости (также область реализуемых или достижимых режимов).
	
	Для рассмотрения необратимых процессов к двум вышеуказанным балансам необходимо добавить еще одно балансовое уравнение для фактора, характеризующего необратимость процесса. Этим фактором является энтропия. В уравнение энтропийного баланса входит величина $\sigma$, называемая диссипацией или производством энтропии. Производство энтропии всегда неотрицательно, при этом нулевое значение $\sigma$ соответствует обратимым процессам. Неравенство $\sigma > 0$ формирует дополнительное ограничение на область реализуемости процесса.
	
	\subsubsection{Примеры процессов в термодинамических системах}
	
		\paragraph{Непосредственный тепловой контакт двух тел.}	Рассмотрим теплового контакта двух тел, изолированных от внешней среды с различными начальными температурами $T_1$ и $T_2$. Через достаточно большое время их температуры выравниваются до некоторого среднего значения. Чтобы вернуть систему в начальное состояние, необходимо охладить одно тело и нагреть второе. При этом должна быть затрачена некоторая работа $A$, что приведет к изменению состояния окружения. Это свидетельствует о том, что процесс непосредственного теплового контакта необратим.
		
		\paragraph{Тепловой контакт двух тел через идеальную тепловую машину.} Пусть в вышеприведенной системе тела контактируют через идеальную тепловую машину. В ней нет потерь на трение, и она получает тепло от горячего тела при температуре газа в цилиндре сколь угодно близкой к температуре нагревателя, а отдает тепло при температуре газа сколь угодно близкой к температуре холодильника. Продолжительность процесса при этом неограничена, а машина будет извлекать работу до тех пор, пока температуры тел не выровняются. Общая температура системы окажется ниже, чем при непосредственном контакте, а полученной работы хватит для того, чтобы вернуть систему в исходное состояние при наличии неограниченного времени. Процесс контакта через посредника --- идеальную тепловую машину обратим.
	
	\subsection{Оптимизационная термодинамика}
	
		\subsubsection{Постановки задач оптимизационной термодинамики}
		
		Каждая неизолированная термодинамическая система  обменивается с окружением энергетическими и материальными потоками. Механизм функционирования системы (кинетика тепло и массообмена, химических реакций и пр.) устанавливает связь между входящими и выходящими потоками. Из выходящих потоков можно выделить или сформировать целевой поток. Его интенсивность представляет собой производительность системы. Входящие потоки формируют поток затрат. Для тепловой машины целевой поток – мощность. Поток затрат – теплота, отбираемая от горячего источника.
		
		\paragraph{Задача 1. О максимально возможной производительности.} Задача о такой форме цикла тепловой машины, получающей теплоту от источника бесконечной емкости с температурой $T_+$ и отдающей теплоту источнику с температурой $T_-$, при которой мощность машины была бы максимальной. Также возникает вопрос о том, для каких систем производительность ограничена сверху, а для каких, увеличивая поток затрат, можно сделать производительность сколь угодно большой.
		
		\paragraph{Задача 2.} Задача о максимальном значении целевого потока системы из двух или более резервуаров и рабочего тела, контактирующего в стационарном режиме или поочередно с каждым из резервуаров и вырабатывающего целевой поток. Также ставится вопрос о нахождении КПД такой системы.
				
		\paragraph{Задача 3.} Аналогична задаче 2, но вместо резервуаров (источников бесконечной емкости) используются источники конечной емкости. Это задача о получении максимальной работы	в замкнутой термодинамической системе за фиксированное время. Она совпадает с задачей о вычислении эксергии системы в том случае, когда продолжительность процесса не ограничена.
		
		\paragraph{Задача 4.} Задача о минимуме прироста энтропии и соответствующей организации термодинамических процессов при заданной средней интенсивности потоков (процессы минимальной диссипации).
		
		\paragraph{Задача 5.} Задача о построении области реализуемости процесса в пространстве, по осям которого отложены интенсивности потоков, при условии необратимости процесса. Решение этой задачи показывает, какими могут быть  требования к системе и как эти требования  связаны с ограничениями на кинетические коэффициенты и продолжительность процесса.
		
		В данной работе получены некоторые важные решения задач 4 и 5 для процессов теплообмена и разделения смесей.
		
		\subsubsection{Методология оптимизационной термодинамики}
		
		Первым шагом в решении задач, приведенных выше и им подобных является составление балансовых соотношений по веществу, энергии и энтропии, с включением в последний из балансов величины производства энтропии $\sigma$. В любой реальной системе $\sigma \geq \sigma_{min}$, что сужает множество реализуемости.
		
		На втором шаге из балансовых уравнений получают зависимость между каким-либо показателем эффективости системы (например, целевым потоком) и диссипацией $\sigma$.
		
		Третьим шагом является решение задачи о такой организации процессов, для которой диссипация минимальна при заданных ограничениях. В сложной системе диссипация аддитивно зависит от диссипации в каждом из элементарных процессов, в каждой подсистеме. Оптимальная организация процессов в сложной системе сводится к согласованию друг с другом отдельных процессов минимальной диссипации.
		
		При составлении термодинамических балансов входящие потоки считают положительными, а выходящие отрицательными. Различат диффузионные и конвективные потоки. Диффузионный поток, в отличие от конвективного, зависит от различия между интенсивными переменными системы в точке, куда он входит или откуда выходит и интенсивными переменными окружения. Диффузионные потоки в дальнейшем выделяются индексом $d$.
		
		\subsubsection{Основные результаты оптимизационной термодинамики}

\section{Множество реализуемых режимов колонны бинарной ректификации}

	\subsection{Термическое разделение смесей и бинарная ректификация}
	
	Процессы разделения смесей --- важный класс процессов, рассматриваемых в термодинамике. Существует два основных вида разделения: механическое и термическое. К первому виду относится разделение при помощи центрифуг и мембран, когда поток затрат является потоком механической энергии, а ко второму --- перегонка и ректификация, в которых потоком затрат является поток тепла.

	\subsection{Математическая модель процесса ректификации двухкомпонентной смеси}
	
	Простота процесса бинарной ректификации позволяет использовать для его описания более точную математическую модель чем для процессов разделения смесей из большего числа компонентов.
	
	\paragraph{Смесь.} В этой и последующих главах будет предполагаться, что объектом разделения является зеотропная смесь бесконечно близкая к идеальному раствору (для нее выполняется первый закон Рауля о том, что парциальное давление насыщенного пара компонента раствора прямо пропорционально его мольной доле в растворе, причём коэффициент пропорциональности равен давлению насыщенного пара над чистым компонентом). В случае бинарной ректификации смесь состоит из двух компонентов с условными номерами $0$ и $1$, называемыми, соответственно, низкокипящим и высококипящим. Компоненты обладают следующими технологическими параметрами: концентрация $x_i$, температура кипения $T_i$, мольная теплота парообразования $r_i$. Технологические параметры каждого компонента имеют соответствующий индекс. Из физического смысла величин следуют условия:
	\[
	x_0 + x_1 = 1,
	\]
	\[
	T_0 < T_1
	\]
	
	\paragraph{Колонна.} Колонна бинарной ректификации, являющаяся элементарным аппаратом в системе разделения смесей во всех рассматриваемых случаях моделью с тремя основными технологическими параметрами: коэффициентами теплопереноса в кубе $\beta_B$, в дефлегматоре $\beta_D$ и коэффициентом массопереноса $k$.
	
	\paragraph{Постановка задачи} Задачам  оптимальной организации процесса ректификации посвящено огромное число исследований (см. \cite{PlnvskNklv}, \cite{Gelp}, \cite{Aleksandr1}, \cite{PetlSer}, \cite{Pavlov}, \cite{Kafar}   и др.). При этом (см. \cite{Gelp}, стр.540) методы расчета <<базируются на диаграммах фазового равновесия и материальном балансе процесса ректификации, но совершенно игнорируют его
кинетику и гидродинамическую обстановку в аппаратах. Для восполнения этого пробела введены коэффициенты полезного действия, которые не поддаются точному теоретическому определению, а могут быть приближенно вычислены по эмпирическим формулам или заимствованы из практики>>.

К этой цитате из учебника Н.И. Гельперина нужно добавить, что очень важным при расчете  колонны, как это показано далее, является учет необратимости процесса теплообмена при подаче теплоты в куб и отборе ее в дефлегматоре.
  
  Для выбора последовательности разделения многокомпонентных смесей используют без какого бы то ни было обоснования эвристические правила  (см.\cite{Kafar} стр. 289), например: <<В первую очередь надо отделять самый легколетучий компонент>> или <<В первую очередь надо отделять компонент с наибольшей концентрацией>>. 
  
  Затраты энергии на разделения зависят от температур кипения компонентов, составов потоков на входе и на выходе процесса. Эти составы определят обратимую изотермическую работу разделения, которая не зависит от того, в каком порядке разделяют смесь. Таким образом при учете только обратимых факторов порядок разделения определен температурами кипения компонентов. Учет необратимости позволяет оценить влияние порядка разделения как на затраты теплоты при теплообмене, так и на потери теплоты в процессе массопереноса и выбрать последовательность разделения, соответствующий минимуму суммарных затрат теплоты в каскаде колонн. 
     
     В \cite{Berry}, \cite{TsGrig} учет необратимости процесса сделан через кинетику тепло- и массообмена, что позволяет проследить влияние кинетических факторов на предельные возможности колонны (производительность, расход теплоты), найти оценку неизбежных необратимых затрат энергии и  организовать процесс разделения так, чтобы эти затраты были возможно меньше. Однако в этих работах необратимость массопереноса учитывалась в алгоритмической фрме, что не позволяло в явном виде представить  зависимость предельной производительности от расхода теплоты и кинетических факторов.   
Ниже такая зависимость получена и показано, что на ее базе может быть предложен алгоритм для выбора последовательности разделения трехкомпонентных смесей.
     
      Так как в дальнейшем мы используем оценки сверху возможностей колонны, то допущения, упрощающие расчет и расширяющие применимость результатов, сделаны так, чтобы каждое из них не увеличивало необратимость процессов. Только в этом случае можно утверждать, что показатели реальной колонны не превосходят  найденных. Полученные таким образом оценки гораздо ближе к истине, чем те, что построены на базе обратимых процессов. Кроме того, такой показатель, как предел производительности колонны с заданными коэффициентами тепло и массопереноса с использованием обратимых оценок  получить вообще нельзя.    
  
Первоначально запишем соотношения, определяющие множество допустимых режимов  ректификации  бинарной смеси, и покажем, что границу этой области можно  параметризовать квадратичной функцией. Приведем связь коэффициентов параметризации (\textit{характеристических коэффициентов}) с составом смеси и кинетическими константами. Затем с использованием этих результатов для смеси из трех компонентов решим задачу о выборе порядка разделения и построении области реализуемых режимов каскада с использованием параметризованного представления. Связи между коэффициентами параметризации и технологическими параметрами позволяют выразить искомые условия через состав смеси и кинетику процессов.

	\subsection{Множество достижимости колонны бинарной ректификации}

	\subsection{Кинетические коэффициенты процесса бинарной ректификации}
		\subsubsection{Коэффициенты теплопереноса}
		
		\subsubsection{Коэффициент массопереноса}

\section{Множество реализуемых режимов каскада из двух колонн бинарной ректификации и оптимальная последовательность разделения трехкомпонентной смеси}

	\subsection{Математическая модель процесса ректификации трехкомпонентной смеси}
	
	\subsection{Выбор порядка разделения трехкомпонентной смеси}
	
	\subsection{Множество достижимости каскада из двух колонн бинарной ректификации}
		\subsection{Особые случаи}
	

\section{Множество реализуемых режимов каскада из произвольного числа колонн бинарной ректификации и оптимальная последовательность разделения многокомпонентной смеси}

	\subsection{Математическая модель процесса ректификации многокомпонентной смеси}
	
	\subsection{Выбор порядка разделения многокомпонентной смеси}
	
	\subsection{Множество достижимости каскада из произвольного числа колонн бинарной ректификации}

\section{Множество реализуемых режимов процессов теплообмена}
	\subsection{Математическая модель процесса теплообмена}
	
	\subsection{Множество достижимости процесса двухпоточного теплообмена}
	
	\section*{Список литературы}
\begin{enumerate}
\bibitem{PlnvskNklv}
{\it Плановский А.Н., Николаев П.И.} Процессы и аппараты химической
и нефтехимической технологии. Учебник для вузов. М.: Химия, 1987.
\bibitem{Gelp}
{\it Гельперин Н.И.} Основные процессы и аппараты химической технологии. М.:
Химия, 1981.
\bibitem{Aleksandr1}
{\it Александров И.А.} Ректификационные и абсорбционные аппараты. М.: Химия. 1978.
\bibitem{PetlSer}
{\it Петлюк Ф.Б., Серафимов Л.А.} Многокомпонентная ректификация. Теория и расчет. М.:Химия, 1983.
\bibitem{Pavlov}
{\it Павлов К.Ф., Романков П.Г., Носков А.А.} Примеры и
задачи по курсу процессов и аппаратов химической технологии.
Ленинград: Химия, 1976.
\bibitem{Kafar}
{\it  Кафаров В.В., Мешалкин В.П., Перов В.Л.}Математические основы автоматизированного проектирования химических производств. М.: Химия, 1979. 
\bibitem{Berry}
{\it Berry R.S., Kasakov V.A., Sieniutycz S., Tsirlin A.M.} Thermodynamic Optimization of Finite Time Processes. // Chichester: John Wiley and Sons, 1999.
\bibitem{TsGrig}
{\it Tsirlin A.M., Grigorevsky I.N.} Thermodynamical estimation of the limit capacity of irreversible binary distillation -  J. Non-Equilibrium Thermodynamics, 2010, V.35 p.213-233
\bibitem{Amelkin}
{\it Амелькин С.А., Бурцлер Й.М., Хоффман К.Х., Цирлин А.М.}Оценка 
предельных возможностей процессов разделения. //Теорет.осн. хим. 
технологии. Т. 35, № 3. 2001 г.
\bibitem{Tsirlin1}
{\it Tsirlin A. M., Kazakov V.A.,} Irreversible work of separation and heatdriven separation // J.Phys.Chem. B 2004. V.108. P 6035--6042.
\bibitem{Tsirlin2}
{\it Цирлин А.М., Вясилева Э.Н., Романова Т.С.} Выбор  термодинамически оптимальной последовательности разделения многокомпонентных смесей и распределения поверхностей тепло и массообмена. // Теор. осн. хим. технологии,Т.43, №3, 2009 г.
\bibitem{Tsirlin3}
{\it Цирлин А.М., Зубов Д.А.,Барбот А.} Учет фактора необратимости в процессе бинарной ректификации.// Теорет.осн. химической технологии. Т.40, №2, 2006.
\bibitem{Tsirlin08}
{\it Цирлин А.М., Ахременков А.А., Григоревский И.Н.} Минимальная необратимость, оптимальное распределение поверхности и тепловой нагрузки теплообменных систем.// Теор. осн. хим. технологии,т.42,№1, 2008г.
%\bibitem{TMAK}
%{\it Tsirlin A.M., Mironova V.A., Amelkin S.A., Kazakov V.A.}
%Finite-Time Thermodynamics: Conditions of Minimal Dissipation for
%Thermodynamic Process with Given Rate // Physical Review E. 1998.
%V.58. №1. P. 215.
\end{enumerate}

\end{document}